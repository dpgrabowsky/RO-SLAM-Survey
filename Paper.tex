\documentclass[conference]{IEEEtran}


\begin{document}
	\title{I am a title (RO SLAM Methods/Implementation Survey) }

	\author{David Grabowsky}
	
	\markboth{IEEE Transactions On xxxl, Vol. XX, No. Y, Month 2018}{Grabowsky: RO SLAM}
	
	
	
	\maketitle
	
	
\begin{abstract}

	
	This is my abstract, there are many like it, but this one is mine. Will fill this in once paper is written ****
	
	An survey of current the current implementations and methodologies used for solving the range only systematic and localization problem (RO-SLAM). 

\end{abstract}
	
	
	
\section{Introduction} 

\section{range only sensors/tech}
\section{Methods...}
\subsection{EKF}
%breif description (1-2 paragraphs)
%paragraphs 
The Extended Kalman Filter (EKF) is one of the most popular and widespread methods used to solve the SLAM problem. It is used to overcome the assumption of linear state transitions and measurements, which are rarely seen in practical environments. \cite{Thrun2002}. The derivation of the EKF is very well documented, as such, detailed explanations can be seen from a variety of sources such as: \cite{Thrun2002} \cite{Ribeiro2004} \cite{Haykin2001}. When used to solve SLAM the map applied to the EKF is a feature based map, meaning it is composed of observable features (landmarks) which can be distinguished between during re-observation.\cite{Thrun2002} This distinction becomes important when examining range only SLAM (RO-SLAM) where only the range to a landmark, or the range between landmarks, is known. This restriction means that landmarks need to be manufactured, such as those discussed in section II, and physically placed in an environment.

Multipathing and noise presents major obstacle to any radio frequency based RO-SLAM solutions. Research and experimentation has been conducted to determine how these effects can be mitigated without the use of specialized equipment. Fabreese presents a pre-filtering algorithm to be applied to range measurment before being used in the EKF in order to reduce these effects \cite{Fabresse2014}. The author also validates this method through indoor and outdoor experimentation with an Unmanned Aircraft System (UAS). \cite{Fabresse2016}. 

%fabreese. 
%1. 2013 presents EKF slam method for ariel vehicles 
%2. 2014 introduces prefiltering to measurments for previous 
%3. 2014a integrates visual landmakrs as well
%4  2015 decentralized multiple arieal vehicles
%5  2016 specifically unmaned ariel, experimental validation.


Vallicrosa presents a solution utilizing a Sum of Gaussian (SOG) filter for a single range only beacon. The filter makes use the EKF by representing each Gaussian in the SOG as a complete EKF, results are given in a simulated environment.\cite{Vallicrosa2015}

One of the complications with RO-SLAM involves the issue of determining the location of landmarks when their initial placement is unknown, the EKF is especially sensitive to poorly initialized landmarks. 




\subsection{Graph SLAM}
\subsection{Particle}
\subsection{Graph}
\subsection{Fast}

\section{Conclusion}


	\bibliographystyle{IEEEtran}
	\bibliography{references}
	
\end{document}

