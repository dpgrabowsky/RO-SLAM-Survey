\documentclass[conference]{IEEEtran}


%just a test references section

%use mendaly 
\begin{filecontents*}{references.bib}
	@article{Khoe:1994:CML:2288694.2294265,
		author = {Khoe, G. -D.},
		title = {Coherent multicarrier lightwave technology for flexible capacity networks},
		journal = {Comm. Mag.},
		issue_date = {March 1994},
		volume = {32},
		number = {3},
		month = mar,
		year = {1994},
		issn = {0163-6804},
		pages = {22--33},
		numpages = {12},
		url = {http://dx.doi.org/10.1109/35.267438},
		doi = {10.1109/35.267438},
		acmid = {2294265},
		publisher = {IEEE Press},
		address = {Piscataway, NJ, USA},
	}
\end{filecontents*}



\begin{document}
	\title{I am a title (RO SLAM Methods/Implementation Survey) }

	\author{David Grabowsky}
	
	\markboth{IEEE Transactions On xxxl, Vol. XX, No. Y, Month 2018}{Grabowsky: RO SLAM}
	
	
	
	\maketitle
	
	
\begin{abstract}

	
	This is my abstract, there are many like it, but this one is mine. Will fill this in once paper is written ****
	
	An survey of current the current implementations and methodologies used for solving the range only systematic and localization problem (RO-SLAM). 

\end{abstract}
	
	
	
\section{Introduction} 

\section{range only sensors/tech}
\section{Methods...}
\subsection{EKF}
%breif description (1-2 paragraphs)
%paragraphs 
The Extended Kalman Filter (EKF) is one of the most popular and widespread methods used to solve the SLAM problem. The map in EKF is a feature based map, meaning it is composed of observable features (landmarks) which can be distinguished between during re-observation.  
%Probabilistic robotics page 312
Due to this distinction  	
	
\subsection{Graph SLAM}
\subsection{Particle}
\subsection{Graph}
\subsection{Fast}

\section{Conclusion}

	
	\bibliographystyle{ieeetran}
	\bibliography{references}
	
\end{document}

